% Options for packages loaded elsewhere
\PassOptionsToPackage{unicode}{hyperref}
\PassOptionsToPackage{hyphens}{url}
\PassOptionsToPackage{dvipsnames,svgnames*,x11names*}{xcolor}
%
\documentclass[
  12pt,
  russian,
  a4paper,
  12pt,
  oneside,
  openany]{book}
\usepackage{lmodern}
\usepackage{amssymb,amsmath}
\usepackage{ifxetex,ifluatex}
\ifnum 0\ifxetex 1\fi\ifluatex 1\fi=0 % if pdftex
  \usepackage[T1]{fontenc}
  \usepackage[utf8]{inputenc}
  \usepackage{textcomp} % provide euro and other symbols
\else % if luatex or xetex
  \usepackage{unicode-math}
  \defaultfontfeatures{Scale=MatchLowercase}
  \defaultfontfeatures[\rmfamily]{Ligatures=TeX,Scale=1}
  \setmainfont[]{DejaVu Serif}
  \setmonofont[Scale=0.8]{DejaVu Sans Mono}
\fi
% Use upquote if available, for straight quotes in verbatim environments
\IfFileExists{upquote.sty}{\usepackage{upquote}}{}
\IfFileExists{microtype.sty}{% use microtype if available
  \usepackage[]{microtype}
  \UseMicrotypeSet[protrusion]{basicmath} % disable protrusion for tt fonts
}{}
\makeatletter
\@ifundefined{KOMAClassName}{% if non-KOMA class
  \IfFileExists{parskip.sty}{%
    \usepackage{parskip}
  }{% else
    \setlength{\parindent}{0pt}
    \setlength{\parskip}{6pt plus 2pt minus 1pt}}
}{% if KOMA class
  \KOMAoptions{parskip=half}}
\makeatother
\usepackage{xcolor}
\IfFileExists{xurl.sty}{\usepackage{xurl}}{} % add URL line breaks if available
\IfFileExists{bookmark.sty}{\usepackage{bookmark}}{\usepackage{hyperref}}
\hypersetup{
  pdftitle={Devdoc-Swissknife-Ru},
  pdfauthor={Николай Гнитеев},
  pdflang={ru},
  colorlinks=true,
  linkcolor=Maroon,
  filecolor=Maroon,
  citecolor=Blue,
  urlcolor=Blue,
  pdfcreator={LaTeX via pandoc}}
\urlstyle{same} % disable monospaced font for URLs
\usepackage{longtable,booktabs}
% Correct order of tables after \paragraph or \subparagraph
\usepackage{etoolbox}
\makeatletter
\patchcmd\longtable{\par}{\if@noskipsec\mbox{}\fi\par}{}{}
\makeatother
% Allow footnotes in longtable head/foot
\IfFileExists{footnotehyper.sty}{\usepackage{footnotehyper}}{\usepackage{footnote}}
\makesavenoteenv{longtable}
\usepackage{graphicx}
\makeatletter
\def\maxwidth{\ifdim\Gin@nat@width>\linewidth\linewidth\else\Gin@nat@width\fi}
\def\maxheight{\ifdim\Gin@nat@height>\textheight\textheight\else\Gin@nat@height\fi}
\makeatother
% Scale images if necessary, so that they will not overflow the page
% margins by default, and it is still possible to overwrite the defaults
% using explicit options in \includegraphics[width, height, ...]{}
\setkeys{Gin}{width=\maxwidth,height=\maxheight,keepaspectratio}
% Set default figure placement to htbp
\makeatletter
\def\fps@figure{htbp}
\makeatother
\setlength{\emergencystretch}{3em} % prevent overfull lines
\providecommand{\tightlist}{%
  \setlength{\itemsep}{0pt}\setlength{\parskip}{0pt}}
\setcounter{secnumdepth}{5}
\ifxetex
  % Load polyglossia as late as possible: uses bidi with RTL langages (e.g. Hebrew, Arabic)
  \usepackage{polyglossia}
  \setmainlanguage[]{russian}
\else
  \usepackage[shorthands=off,main=russian]{babel}
\fi
\usepackage[]{natbib}
\bibliographystyle{apalike}

\title{Devdoc-Swissknife-Ru}
\author{Николай Гнитеев}
\date{2022-05-15}

\begin{document}
\maketitle

{
\hypersetup{linkcolor=}
\setcounter{tocdepth}{2}
\tableofcontents
}
\listoftables
\listoffigures
\hypertarget{ux432ux441ux442ux443ux43fux43bux435ux43dux438ux435}{%
\chapter*{Вступление}\label{ux432ux441ux442ux443ux43fux43bux435ux43dux438ux435}}


Этот проект демонстрирует подход к созданию документации на разработку с помощью \href{https://rmarkdown.rstudio.com/}{R Markdown} и \href{https://kroki.io/}{Kroki}.

Всё то, что описано в данном в проекте, можно было делать и прежде, но я попытался увязать вместе эти компоненты так, чтобы можно было сосредоточить усилия на содержательной части документации.

Данный подход позволяет создавать документацию в виде файлов PDF, презентаций, сайта документтации по шаблону gitbook и в некоторых других форматах, используя для этого текстовые файлы с синтаксисом markdown (Pandoc flavor) и дополнительными вставками кода на языке R, а так же текстовые описания разного рода диаграмм.

Целью этого проекта является создание документации, максимально используя для этого текстовый формат, в т.ч. для описания графических диаграмм, но при этом без излишних переусложнений.

\begin{quote}
На данный момент рускоязычная документация устарела и нуждается в переработке. Посмотрите \href{devdoc-swissknife-en/index.html}{англоязычный вариант} пока работа ещё не завершена.
\end{quote}

  \bibliography{book.bib,packages.bib}

\end{document}
